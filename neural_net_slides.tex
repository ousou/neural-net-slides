\documentclass{beamer}
\usetheme{Bergen}
\usecolortheme{beaver}
\usepackage{amsmath}
\usepackage[utf8]{inputenc}

\def\insertauthorindicator{Who?}% Default is "Who?"
\def\insertinstituteindicator{From?}% Default is "From?"
\def\insertdateindicator{When?}% Default is "When?"

\title{Neural networks}
\subtitle{Architectures and training tips}

\author{Sebastian Bj{\"o}rkqvist}
\institute{IPRally Technologies}

\date[09.01.2019]{09.01.2019}

\newcommand{\kur}{\protect\textit}
\newcommand{\bol}{\protect\textbf}

\begin{document}

\frame{\titlepage}

  \begin{frame}
    \frametitle{Why neural networks?}  
    
   	\begin{itemize}
		\item Can approximate any function \cite{hornik}
		\item May learn to respond to unexpected patterns
		\item Useful especially when the amount of data is large
		\item Less need for feature engineering compared to traditional ML methods
	\end{itemize}
  \end{frame}

   
   \begin{frame}
   	\frametitle{References}
   	\begin{thebibliography}{Hornik, 1991}

  \bibitem[Nielsen, 2015]{nielsen} Nielsen, Michael A. {\em Neural Networks And Deep Learning}. Determination Press, 2015. \url{http://neuralnetworksanddeeplearning.com/}
  
  \bibitem[Hornik, 1991]{hornik} Hornik, Kurt. {\em Approximation Capabilities of Multilayer Feedforward Networks}. Neural Networks, 4(2), 251--257, 1991.
    
	\end{thebibliography}   
   \end{frame}  

\end{document}
