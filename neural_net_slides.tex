\documentclass{beamer}
\usetheme{Bergen}
\usecolortheme{beaver}
\usepackage{amsmath}
\usepackage[utf8]{inputenc}

\def\insertauthorindicator{Who?}% Default is "Who?"
\def\insertinstituteindicator{From?}% Default is "From?"
\def\insertdateindicator{When?}% Default is "When?"

\title{Neural networks}
\subtitle{Architectures and training tips}

\author{Sebastian Bj{\"o}rkqvist}
\institute{IPRally Technologies}

\date[09.01.2019]{09.01.2019}

\newcommand{\kur}{\protect\textit}
\newcommand{\bol}{\protect\textbf}

\begin{document}

\frame{\titlepage}

  \begin{frame}
    \frametitle{Why neural networks?}  
    
   	\begin{itemize}
		\item Can approximate any function \cite{hornik}
		\item May learn to respond to unexpected patterns
		\item Useful especially when the amount of data is large
		\item Less need for feature engineering compared to traditional ML methods
	\end{itemize}
  \end{frame}

  \begin{frame}
    \frametitle{Mit{\"a} on informaatio?}   
    \Large Algoritmien ollessa kyseess{\"a} \bol{Churchin-Turingin teesi} kertoo mink{\"a}laiset asiat ovat laskettavissa. \\
    
    \vspace{8mm}
    
    \onslide<2->{Mill{\"a} tavalla voidaan vastaavasti \\ m{\"a}{\"a}ritell{\"a} informaatio?}
    
  \end{frame}
  
 
  \begin{frame}
    \frametitle{Merkkijonon informaatiom{\"a}{\"a}r{\"a}}   
	
	\Large Haluamme m{\"a}{\"a}ritell{\"a} jonon informaatiom{\"a}{\"a}r{\"a}n ja verrata sit{\"a} \\ jonon pituuteen.
    
	\vspace{8mm}
   
    Informaatiom{\"a}{\"a}r{\"a} on siis merkkijono.
  \end{frame}  
  
  \begin{frame}
    \frametitle{Merkkijonon lyhin kuvaus}   
    \Large Olkoon $x$ merkkijono bin{\"a}{\"a}rimuodossa. Jonon $x$ \bol{lyhin kuvaus} $d(x)$ on lyhin merkkijono $\langle M,w \rangle$, jolle p{\"a}tee ett{\"a} Turingin kone \kur{M}  palauttaa merkkijonon $x$ sy{\"o}tteell{\"a} $w$.  \\  
    \vspace{10px}
    
    M{\"a}{\"a}rittelemme jonon $\langle M,w \rangle$ konkatenoituna merkkijonona \kur{M}\kur{w}.
    
  \end{frame}  
 
  \begin{frame}
    \frametitle{Kolmogorov-kompleksiteetti}   
    
    \Large Jonon $x$ \bol{Kolmogorov-kompleksiteetti} \\ $K(x)$ on jonon lyhimm{\"a}n kuvauksen pituus, toisin sanoen

    \begin{center}
    		\huge $K(x) = |d(x)|$
    \end{center}
  \end{frame}   
  
   \begin{frame}
    \frametitle{Kolmogorov-kompleksiteetin perustuloksia}   
    \Large On olemassa vakiot $c_{1}$ ja $c_{2}$, joille p{\"a}tee
    
    \begin{enumerate}
    		\item $K(x) \: \: \le |x| + c_{1}$
    		\item $K(xx) \le K(x) + c_{2}$
    \end{enumerate}
    
	kaikilla merkkijonoilla $x$. 
    
   \end{frame}   
   
   \begin{frame}
    \frametitle{Konkatenoidun jonon kompleksiteetti} 
	\Large Kahden mielivaltaisen jonon konkatenaation kompleksiteetille voimme helposti todistaa seuraavan yl{\"a}rajan:
	
	\vspace{7mm}
	
	\Large On olemassa vakio $c$, jolle p{\"a}tee
	
	\begin{itemize}
		\item $K(xy) \le 2\cdot K(x) + K(y) + c$
	\end{itemize}
	
	\vspace{2mm}	
	
	kaikilla merkkijonoilla $x$ ja $y$.
   \end{frame}  
   
   \begin{frame}
    \frametitle{Konkatenoidun jonon kompleksiteetti}    
    
    \Large Todistuksen idea: Tuplataan kuvauksen $d(x)$ bitit, ja konkatenoidaan per{\"a}{\"a}n kuvaus $d(y)$. V{\"a}liin lis{\"a}t{\"a}{\"a}n erotin 01.

	\vspace{3mm}    
    \Large \onslide<2->{Esimerkiksi jonoilla $d(x) = 0110110$ ja $d(y) = 1110001$ saadaan }
  
    \onslide<3->{\[
\underbrace{00111100111100}_{\text{d(x) tuplattuna}}01\underbrace{1110001}_\text{d(y)}
\]}
   \end{frame}    
   
   \begin{frame}
    \frametitle{Konkatenoidun jonon kompleksiteetti}
   
   \Large Voimme saavuttaa my{\"o}s yl{\"a}rajan 
   
   	\begin{itemize}
		\item $K(xy) \le 2\cdot log_{2}(K(x)) + K(x) + K(y) + c$,
	\end{itemize}
	
	\onslide<2->{mutta osoittautuu mahdottomaksi saavuttaa raja}
	
	\begin{itemize}
		\item<2-> $K(xy) \le K(x) + K(y) + c$.
	\end{itemize}
   \end{frame}    
    
   \begin{frame}
    \frametitle{Kolmogorov-kompleksiteetin optimaalisuus}
    
    \Large Turingin koneen vaihtaminen joksikin muuksi funktioksi m{\"a}{\"a}ritelm{\"a}ss{\"a} ei olennaisesti muuta merkkijonon kompleksiteettia.
    
    \vspace{8mm}
    
    T{\"a}m{\"a} johtuu siit{\"a}, ett{\"a} voimme simuloida kaikkia algoritmeja Turingin koneella.
    
   \end{frame}    
   
   \begin{frame}
    \frametitle{Tiivistyv{\"a}t merkkijonot}
    
    \Large Merkkijono $x$ on \bol{c-tiivistyv{\"a}}, jos
    	\begin{center}
    		$K(x) \leq |x| - c$.
    	\end{center}
    	
    Jos merkkijono ei ole 1-tiivistyv{\"a}, on se \bol{tiivistym{\"a}t{\"o}n}.
   
   \end{frame}  
   
   \begin{frame}
    \frametitle{Tiivistym{\"a}t{\"o}n merkkijono?}
    
    \Large Onko olemassa tiivistym{\"a}tt{\"o}mi{\"a} merkkijonoja?
    
    \vspace{3mm}

	\onslide<2->{Pituutta n olevia merkkijonoja on $2^{n}$ kappaletta,}
	\onslide<2->{mutta t{\"a}t{\"a} lyhyempi{\"a} merkkijonoja on vain $2^{n}-1$ kappaletta.\\}
	\vspace{3mm}
	\onslide<2->{Ainakin yksi pituutta n oleva jono on siis tiivistym{\"a}t{\"o}n.}
   
   \end{frame}
     
   \begin{frame}
    \frametitle{Kompleksiteetin m{\"a}{\"a}ritt{\"a}minen}
    
    \Large Brute force-algoritmi ei toimi, sill{\"a} saatamme ajautua ikuiseen silmukkaan. 
    
	\vspace{3mm}    
    
    \onslide<2->{Yleist{\"a} algoritmia kompleksiteetin m{\"a}{\"a}ritt{\"a}miselle ei ole. \\}
    
    \vspace{3mm}
    \onslide<2->{Ongelma on itseasiassa Turing-yht{\"a}pit{\"a}v{\"a} \\
    \vspace{2mm} 
    pys{\"a}htymisongelman kanssa.}    
   \end{frame}
   
   \begin{frame}
    \frametitle{Tiivistym{\"a}tt{\"o}m{\"a}n jonon ominaisuuksia} 
    
    \Large Yleisesti ei ole mahdollista selvitt{\"a}{\"a}, onko jokin jono tiivistym{\"a}t{\"o}n. \\
	\vspace{3mm}   
	\Large Tiivistym{\"a}tt{\"o}mill{\"a} jonoilla on paljon samoja ominaisuuksia kuin satunnaisilla jonoilla. \\   
	\vspace{3mm}
	\Large Jonon lyhin kuvaus $d(x)$ on l{\"a}hes tiivistym{\"a}t{\"o}n.
    
   \end{frame}   
   
   \begin{frame}
    \frametitle{Yhteenveto}
    
   \begin{itemize}
	\Large \item Kolmogorov-kompleksiteetti kertoo merkkijonon informaatiosis{\"a}ll{\"o}n
	\item Kompleksiteetti ei olennaisesti riipu k{\"a}ytetyst{\"a} laskentamallista
	\item Tiivistym{\"a}tt{\"o}m{\"a}t jonot ovat satunnaisjonojen kaltaisia
	\item Kompleksiteettia ei yleens{\"a} voida m{\"a}{\"a}ritt{\"a}{\"a}
   \end{itemize}
   \end{frame}   
   
   \begin{frame}
   	\frametitle{References}
   	\begin{thebibliography}{Hornik, 1991}

  \bibitem[Nielsen, 2015]{nielsen} Nielsen, Michael A. {\em Neural Networks And Deep Learning}. Determination Press, 2015. \url{http://neuralnetworksanddeeplearning.com/}
  
  \bibitem[Hornik, 1991]{hornik} Hornik, Kurt. {\em Approximation Capabilities of Multilayer Feedforward Networks}. Neural Networks, 4(2), 251--257, 1991.
    
	\end{thebibliography}   
   \end{frame}  

\end{document}
